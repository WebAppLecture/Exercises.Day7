\Exercise{Eine einfache Notizen App}
%
\par Durch Ausnutzen des \jvar{localStorage} Objekts kann man sehr schön
beliebige Objekte (über JSON – bis zu einer bestimmten Größe) speichern.
Erstellen Sie ein Formular, das folgende Felder besitzt:
%
\begin{itemize}
\item
Titel
\item
Beschreibung
\item
Priorität
\end{itemize}
%
\par Mit einem Klick auf \emph{Hinzufügen} sollen Sie Einträge erstellen
(Erstellen eines neuen Storage Eintrags, inkl. Erstelldatum und letztem
Änderungsdatum – alles als serialisiertes JSON Objekt). Die erstellten Einträge
sollen in einer seperaten Liste (\htag{aside}) angezeigt werden und sich bei
Klick in das Formular laden. Anstelle des \emph{Hinzufügen} Buttons soll dann
ein \emph{Ändern} Button angezeigt werden.
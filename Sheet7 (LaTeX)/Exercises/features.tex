\Exercise{Feature Detection und jQueryUI}
%
\par Verwenden Sie als Grundlage Aufgabe 9 von Übungsblatt 2. Ihre Aufgabe besteht nun darin, das Slider Element sowie den Datepicker durch jQuery UI anzuzeigen. Verwenden Sie daher eine Feature Detection, wie die aus der Vorlesung. Überprüfen Sie ob der Browser bereits die Attributwerte \jvar{type=date} und \jvar{type=range} versteht und führen Sie gegebenenfalls den jQuery UI Code als Fallback ein.
%
\par Bauen Sie des weiteren noch einen Testbutton ein, um auf modernen Browsern manuell in den Fallback-Modus umzuschalten. Dieser Button soll nur angezeigt werden, wenn die Seite ohne Fallback läuft.